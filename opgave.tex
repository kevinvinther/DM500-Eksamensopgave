\documentclass{article}
\usepackage[utf8]{inputenc}
\usepackage{amsmath}
\usepackage{amssymb}
\usepackage{enumerate}

\title{DM500 Eksamensopgaver}
\author{ Kevin Vinther\\kevin20 \and Kasper Halkjær Beider\\kbeid20 \and Necati Øztek}
\date{15. november 2020}

\begin{document}

\maketitle

\newpage

\section{Reeksamen januar 2012 opgave 1}
\textbf{Opgave 1}\\

\textbf{a)} Er \textit{f} en bijektion?\\
\\Nej, da f ikke er injektiv dvs. der findes flere end en x-værdi der rammer den samme y-værdi. Den er heller ikke surjektiv, da der for hvert y i funktionens sekundærmængde ikke findes et x-værdi i definitionsmængden.

\textbf{b)} Har \textit{f} en invers funktion?\\
\\Nej, da den ikke er bijektiv og derfor ikke invertibel.

\textbf{c)} Angiv $\textit{f} + \textit{g}$
\[(f+g)(x)=x^2+x+1+2x-2=x^2+3x-1\]

\textbf{d)} Angiv $\textit{g} \circ \textit{f}$
\[(g \circ f)(x)=g(f(x))=g(x^2+x+1)=2(x^2+x+1)-2=2x^2+2x\]
\section{Reeksamen februar 2015 opgave 1}
I det følgende lader vi $U = {1, 2, 3, ...,15}$ være universet (universal set). \\
Betragt de to mængder
$$A = \{2n | n \in S\} \textrm{ og } B = \{3n+2|n\in S\}$$
hvor $S = \{1, 2, 3, 4\}$.
Angiv samtlige elementer i hver af følgende mængder
\begin{enumerate}[a)]
    \item $A$
    \item $B$
    \item $A \cap B$
    \item $A \cup B$
    \item $A - B$
    \item $\overline{A}$
\end{enumerate}

\textbf{A)}\\
$$A = \{2, 4, 6, 8\}$$
Da $2\cdot1 = 2, 2\cdot2 = 4, 2\cdot 3 = 6, 2\cdot4 = 8$\\

\textbf{B)}\\
$$B = \{5, 8, 11, 14\}$$
Da $3\cdot1+2=5, 3\cdot2+2=8, 3\cdot3+2=11, 3\cdot4=14$\\

\textbf{C)}\\
$$A \cap B = \{2,4,6,8\} \cap \{5,8,11,13\} = \{8\}$$
Da 8 er det eneste element der er gennemgående i begge mængder\\

\textbf{D)}\\
$$A \cup B = \{2,4,6,8\} \cup \{5,8,11,13\} = \{2,4,6,8,5,11,13\}$$
Da $A \cup B$ giver alle elementer i begge mængder.\\

\textbf{E)}\\
$$\{2,4,6,8\} - \{5,8,11,13\} = \{2,4,6\}$$
Da 8 er det eneste element der går igen i den anden mængde, er det det eneste element der bliver fjernet. \\

\section{Reeksamen februar 2015 opgave 2}

\textbf{Opgave 2:}\\
\textbf{a)} Hvilke af følgende udsagn er sande?\\
\textbf{1.} 
\begin{displaymath}
\forall x \epsilon \mathbb{N}: \exists y \epsilon \mathbb{N}: x<y
\end{displaymath}
\textbf{2.}\begin{displaymath}
\forall x \epsilon \mathbb{N}: \exists !y \epsilon \mathbb{N}: x<y
\end{displaymath}
\textbf{3.}\begin{displaymath}
\exists y  \epsilon \mathbb{N}: \forall x  \epsilon  \mathbb{N}: x<y
\end{displaymath}
Første udsagn er sandt. Det kan man konkludere ved at man altid kan sige \(y = x + 1\). Hvilket vil sige at der altid eksiterer en y-værdi der er større end enhver x-værdi. 

Andet udsagn er falskt, da der ikke eksisterer kun et enkelt y-værdi, som er større end enhver x-værdi. 

Tredje udsagn er også falskt. Det kommer af at man ikke kan vælge en y-værdi, hvorom det altid vil gælde at ethvert x-værdi vil være mindre end y-værdien.

\textbf{b)} Negering af udsagn 1. fra spørgsmål a). Negerings tegnet må ikke indgå i udsagnet. 

\begin{displaymath}
\forall x \epsilon \mathbb{N}: \exists y \epsilon \mathbb{N}: x<y
\end{displaymath}
\begin{displaymath}
\neg (\forall x \epsilon \mathbb{N}: \exists y \epsilon \mathbb{N}: x<y)
\end{displaymath}
\begin{displaymath}
\exists x \epsilon \mathbb{N}: \neg \exists y \epsilon \mathbb{N}: x<y
\end{displaymath}
\begin{displaymath}
\exists x \epsilon \mathbb{N}: \forall y \epsilon \mathbb{N}: \neg (x<y)
\end{displaymath}
\begin{displaymath}
\exists x \epsilon \mathbb{N}: \forall y \epsilon \mathbb{N}: x\geq y
\end{displaymath}

\section{Reeksamen februar 2015 opgave 3}
R, S og T er binære relationer på mængden A \(=\lbrace1, 2, 3, 4\rbrace\)

\textbf{Opgave 3:}\\
\textbf{a)} Lad R \(=\lbrace(1,1), (2,1), (2,2), (2,4), (3,1), (3,3), (3,4), (4,1), (4,4)\rbrace\)

Er R en partiel ordning?\\


Definitionen på en partiel ordning er at relationen skal være refleksiv, antisymmetrisk og transitiv. 

\textbf{Refleksiv:} R er refleksiv, da relationen indeholder parrene: \(\lbrace(1,1), (2,2), (3,3), (4,4)\rbrace\)\\

\textbf{Antisymmetrisk:} R er antisymmetrisk, da relationen ikke indeholder nogle par (b,a) når den indeholder (a,b): 

R indeholder parrene = \(\lbrace (2,1), (2,4), (3,1), (3,4), (4,1)\rbrace\), men relationen indeholder ikke parrene = \(\lbrace (1,2), (4,2), (1,3), (4,3), (1,4)\rbrace\). Dermed er relationen antisymmetrisk.  \\

\textbf{Transitiv:} R er transitiv, da der for hvert par (a,b) og (b,c) også er et par (a,c). \\

Eftersom at relationen R opfylder alle tre krav, refleksiv, antisymmetrisk og transitiv, er relationen en partiel ordning. \\

\textbf{b)} Lad S \(=\lbrace(1,2), (2,3), (2,4), (4,2)\rbrace\)
\\Angiv den transitive lukning af S.

\begin{displaymath}
t(S) = S \cup \lbrace(1,3), (1,4), (2,2), (4,3), (4,4)\rbrace
\end{displaymath}
\\
\textbf{c)} Lad T \(=\lbrace(1,1), (1,3), (2,2), (2,4), (3,1), (3,3), (4,2), (4,4)\rbrace\)
\\Angiv T´s ækvivalens-klasser.\\
\\
T´s ækvivlaens-klasser på mængeden A findes ved:
\begin{displaymath}
[ a ] = \lbrace x \epsilon A \mid xTa \rbrace
\end{displaymath}

\begin{displaymath}
[1] = \lbrace 1, 3 \rbrace
\end{displaymath}
\begin{displaymath}
[2] = \lbrace 2, 4 \rbrace
\end{displaymath}
\begin{displaymath}
[3] = \lbrace 1, 3 \rbrace = [1]
\end{displaymath}
\begin{displaymath}
[4] = \lbrace 2, 4 \rbrace = [2]
\end{displaymath}
\\
\textbf{Matricerne for R, S og T:}
\\
\textbf{Matrice for R:}
\begin{displaymath}
\begin{bmatrix}
1 & 0 & 0 & 0\\
1 & 1 & 0 & 1\\
1 & 0 & 1 & 1\\
1 & 0 & 0 & 1\\
\end{bmatrix}
\end{displaymath}
\\
\textbf{Matrice for S:}
\begin{displaymath}
\begin{bmatrix}
0 & 1 & 0 & 0\\
0 & 0 & 1 & 1\\
0 & 0 & 0 & 0\\
0 & 1 & 0 & 0\\
\end{bmatrix}
\end{displaymath}\\
\textbf{Matrice for T:}
\begin{displaymath}
\begin{bmatrix}
1 & 0 & 1 & 0\\
0 & 1 & 0 & 1\\
1 & 0 & 1 & 0\\
0 & 1 & 0 & 1\\
\end{bmatrix}
\end{displaymath}

\end{document}
