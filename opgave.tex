\documentclass{article}
\usepackage[utf8]{inputenc}
\usepackage{amsmath}
\usepackage{amssymb}

\title{DM500 Eksamensopgaver}
\author{ Kevin Vinther \and Kasper Halkjær Beider \and Necati Øztek \and Tue Jensen }
\date{November 2020}

\begin{document}

\maketitle

\newpage

\section{Reeksamen januar 2012 opgave 1}
\textbf{Opgave 1}\\

\textbf{a)} Er \textit{f} en bijektion?\\
\\Nej, da f ikke er injektiv dvs. der findes flere end en x-værdi der rammer den samme y-værdi. Den er heller ikke surjektiv, da der for hvert y i funktionens sekundærmængde ikke findes et x-værdi i definitionsmængden.

\textbf{b)} Har \textit{f} en invers funktion?\\
\\Nej, da den ikke er bijektiv og derfor ikke invertibel.

\textbf{c)} Angiv $\textit{f} + \textit{g}$
\[(f+g)(x)=x^2+x+1+2x-2=x^2+3x-1\]

\textbf{d)} Angiv $\textit{g} \circ \textit{f}$
\[(g \circ f)(x)=g(f(x))=g(x^2+x+1)=2(x^2+x+1)-2=2x^2+2x\]
\section{Reeksamen februar 2015 opgave 1}

\section{Reeksamen februar 2015 opgave 2}

\textbf{Opgave 2:}\\
\textbf{a)} Hvilke af følgende udsagn er sande?\\
\textbf{1.} 
\begin{displaymath}
\forall x \epsilon \mathbb{N}: \exists y \epsilon \mathbb{N}: x<y
\end{displaymath}
\textbf{2.}\begin{displaymath}
\forall x \epsilon \mathbb{N}: \exists !y \epsilon \mathbb{N}: x<y
\end{displaymath}
\textbf{3.}\begin{displaymath}
\exists y  \epsilon \mathbb{N}: \forall x  \epsilon  \mathbb{N}: x<y
\end{displaymath}
Første udsagn er sandt. Det kan man konkludere ved at man altid kan sige \(y = x + 1\). Hvilket vil sige at der altid eksiterer en y-værdi der er større end enhver x-værdi. 

Andet udsagn er falskt, da der ikke eksisterer kun et enkelt y-værdi, som er større end enhver x-værdi. 

Tredje udsagn er også falskt. Det kommer af at man ikke kan vælge en y-værdi, hvorom det altid vil gælde at ethvert x-værdi vil være mindre end y-værdien.

\textbf{b)} Negering af udsagn 1. fra spørgsmål a). Negerings tegnet må ikke indgå i udsagnet. 

\begin{displaymath}
\forall x \epsilon \mathbb{N}: \exists y \epsilon \mathbb{N}: x<y
\end{displaymath}
\begin{displaymath}
\neg (\forall x \epsilon \mathbb{N}: \exists y \epsilon \mathbb{N}: x<y)
\end{displaymath}
\begin{displaymath}
\exists x \epsilon \mathbb{N}: \neg \exists y \epsilon \mathbb{N}: x<y
\end{displaymath}
\begin{displaymath}
\exists x \epsilon \mathbb{N}: \forall y \epsilon \mathbb{N}: \neg (x<y)
\end{displaymath}
\begin{displaymath}
\exists x \epsilon \mathbb{N}: \forall y \epsilon \mathbb{N}: y<x
\end{displaymath}
%Jeg er rimelig sikker på at dette her er rigtigt. 



\end{document}

